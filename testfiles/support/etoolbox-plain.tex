% !TeX root = fsrprotocol3-cls

\ifx\etoolboxplainGUARD\UNDEFINED
\def\etoolboxplainGUARD{etoolbox-plain.tex}

% This is a (re-)implementation of some few commands definied in etoolbox.sty to work under plain TeX

\input latex-plain

% adapted from etoolbox.sty

\def\etb@catcodes{\do\&\do\|\do\:\do\-\do\=\do\<\do\>}
\def\do#1{\catcode\number`#1=\the\catcode`#1\relax}
\edef\etb@catcodes{\etb@catcodes}
\let\do\noexpand

\catcode`\&=3
\catcode`\|=3
\@makeother\:
\@makeother\-
\@makeother\=
\@makeother\<
\@makeother\>

\newtoks\etb@toks

\def\undef#1{%
    \let#1\etb@undefined%
}

% ===== tests ===================================================================================

\long\def\ifstrempty#1{%
    \expandafter\ifx\expandafter&\detokenize{#1}&%
        \expandafter\@firstoftwo%
    \else
        \expandafter\@secondoftwo%
    \fi%
}

\long\def\ifblank#1{% from expl3
    \expandafter\ifx\expandafter\relax\detokenize\expandafter{\@gobble#1?}\relax%
        \expandafter\@firstoftwo%
    \else
        \expandafter\@secondoftwo%
    \fi%
}

\long\def\notblank#1{%
    \expandafter\ifx\expandafter\relax\detokenize\expandafter{\@gobble#1?}\relax%
        \expandafter\@secondoftwo%
    \else%
        \expandafter\@firstoftwo%
    \fi%
}

\long\def\ifundef#1{%
    \ifdefined#1%
        \ifx#1\relax
            \expandafter\expandafter\expandafter\@firstoftwo%
        \else%
            \expandafter\expandafter\expandafter\@secondoftwo%
        \fi%
    \else%
        \expandafter\@firstoftwo%
    \fi%
}

\long\edef\ifdefmacro#1{%
    \noexpand\expandafter\noexpand\etb@ifdefmacro%
    \noexpand\meaning#1\detokenize{macro}:&%
}
\edef\etb@ifdefmacro{%
     \def\noexpand\etb@ifdefmacro##1\detokenize{macro}:##2&%
}
\etb@ifdefmacro{\notblank{#2}}

\long\edef\ifdefparam#1{%
    \noexpand\expandafter\noexpand\etb@ifdefparam%
    \noexpand\meaning#1\detokenize{macro}:->&%
}
\edef\etb@ifdefparam{%
    \def\noexpand\etb@ifdefparam##1\detokenize{macro}:##2->##3&%
}
\etb@ifdefparam{\notblank{#2}}

% ===== expansion ==============================================================================

\long\def\expandonce#1{%
    \unexpanded\expandafter{#1}%
}

% ===== appending & prepending =================================================================

% #1 - \macro
% #2 - code, that will be prepended 
\long\def\preto#1#2{%
%    \ifx#1\etb@undefined%
%        \def#1{#2}%
%    \else%
%        \def\etb@tempb{#2}%
%        \etb@toks\expandafter\expandafter\expandafter{\expandafter\etb@tempb#1}%
%        \edef#1{\the\etb@toks}%
%    \fi%
    \ifundef{#1}%
        {\edef#1{\unexpanded{#2}}}%
        {\edef#1{\unexpanded{#2}\expandonce#1}}%
}

% #1 - \macro
% #2 - code, that will be appended 
\long\def\appto#1#2{%
%    \ifx#1\etb@undefined%
%        \def#1{#2}%
%    \else%
%        \etb@toks\expandafter{#1#2}%
%        \edef#1{\the\etb@toks}%
%    \fi%
    \ifundef{#1}%
        {\edef#1{\unexpanded{#2}}}%
        {\edef#1{\expandonce#1\unexpanded{#2}}}%
}

% #1 - \macro
% #2 - code, that will be fully expanded and prepended 
\long\def\epreto#1#2{%
%    \ifx#1\etb@undefined%
%        \edef#1{#2}%
%    \else%
%        \edef\etb@tempb{#2}%
%        \etb@toks\expandafter\expandafter\expandafter{\expandafter\etb@tempb#1}%
%        \edef#1{\the\etb@toks}%
%    \fi%
    \ifundef{#1}%
        {\edef#1{#2}}%
        {\edef#1{#2\expandonce#1}}%
}

% #1 - \macro
% #2 - code, that will be fully expanded and appeded 
\long\def\eappto#1#2{%
%    \ifx#1\etb@undefined%
%        \edef#1{#2}%
%    \else%
%        \edef\etb@tempb{#2}%
%        \etb@toks\expandafter\expandafter\expandafter{\expandafter#1\etb@tempb}%
%        \edef#1{\the\etb@toks}%
%    \fi%
    \ifundef{#1}%
        {\edef#1{#2}}%
        {\edef#1{\expandonce#1#2}}%
}

% ====== patching mechanism ====================================================================

\protected\def\etb@ifscanable#1{%
\begingroup
    \edef\etb@resrvda{%
        % expands to code, that definies \etb@resrvda as the macro,
        % according to \meaning<macro> as it argument
        % ####1 - prefixes
        % ####2 - parameter text
        % ####3 - replacement text
        \def\noexpand\etb@resrvda####1\detokenize{macro}:####2->####3&{%
            ####1\def\string\etb@resrvda####2{####3}%
        }%
        % expands to code, that simply definies \etb@resrvda as the macro
        \edef\noexpand\etb@resrvda{%
            \noexpand\etb@resrvda\meaning#1&%
        }%
    }%
    \etb@resrvda%
    \makeatletter%
    \endlinechar\m@ne%
    \newlinechar\m@ne%
    \scantokens\expandafter{\etb@resrvda}%
    \expandafter\endgroup\ifx#1\etb@resrvda%
        \expandafter\@firstoftwo%
    \else%
        \expandafter\@secondoftwo%
    \fi%
}

% #1 – macro to test
% #2 – search text
\protected\long\def\etb@ifpattern#1#2{%
\begingroup%
    \edef\etb@resrvda{%
        \def\noexpand\etb@resrvda####1\detokenize{#2}####2&{%
            \endgroup%
            \noexpand\noexpand\noexpand\ifblank{####2}%
        }%
        \edef\noexpand\etb@resrvda{%
            \noexpand\etb@resrvda\expandafter\strip@prefix\meaning#1\detokenize{#2}&%
        }%
        \noexpand\etb@resrvda%
    }%
    \etb@resrvda\@secondoftwo\@firstoftwo%
}

% For the following we define:
% <hash>      ::= the result of \string# (a hash of catcode 12/other)
% <ampersand> ::= &

% checks if the string contains no <hash><hash>
% syntax: \etb@ifhashcheck{string}{true code}{false code}
% #1 - string to test if it contains <hash><hash>
% f1 - true code ; executed if string contains no <hash><hash>
% f2 - false code; executed if string contains    <hash><hash>
\protected\long\def\etb@ifhashcheck#1{%
    \begingroup%
        \edef\etb@resrvda{\detokenize{#1}}%
    \expandafter\endgroup%
    \expandafter\etb@ifhashcheck@i\meaning\etb@resrvda&% {true code}{false code} follows
}

% syntax: \etb@ifhashcheck@i<meaning result>{true code}{false code}
% #1 - result of \meaning\macro
\edef\etb@ifhashcheck@i#1&{%
    \noexpand\expandafter%
    \noexpand\etb@ifhashcheck@ii%
    \noexpand\strip@prefix#1\string#\string#&%
}%

% \etb@ifhashcheck@i is now defined as:
% #1 - result of \meaning\macro
% \def\etb@ifhashcheck@i #1<ampersand>{%
%     \expandafter\etb@ifhashcheck@ii\strip@prefix #1<hash><hash><ampersand>%
% }%

\edef\etb@ifhashcheck@ii{%
    \def\noexpand\etb@ifhashcheck@ii##1\string#\string###2&%
}
\etb@ifhashcheck@ii{\ifblank{#2}}

% \etb@ifhashcheck@ii is now defined as:
% #1 - code in the replacement text before the first <hash><hash>
% #2 - code in the replacement text after the first <hash><hash>
% f1 - true code
% f2 - false code
% \def\etb@ifhashcheck@ii #1<hash><hash>#2<ampersand>{%
%     \ifblank{#2}% {true code}{false code} follows
% }%


\long\def\ifpatchable{%
    \etb@dbg@trce\ifpatchable%
    \begingroup%
        \@makeother\#%
        \@ifstar%
    \etb@ifpatchable@i% closes the group
    \etb@ifpatchable% closes the group
}

\long\def\etb@ifpatchable#1#2{%
    \endgroup%
    \etb@dbg@init#1%
    \ifundef{#1}{%
        \etb@dbg@fail{def}%
        \@secondoftwo%
    }{%
        \etb@dbg@info{def}%
        \ifdefmacro{#1}{%
            \etb@dbg@info{mac}%
            \etb@ifscanable{#1}{%
                \etb@ifhashcheck{#2}{%
                    \etb@dbg@info{tok}%
                    \etb@ifpattern#1{#2}{%
                        \etb@dbg@info{pat}%
                        \etb@dbg@info{pos}%
                        \@firstoftwo%
                    }{%
                        \etb@dbg@fail{pat}%
                        \@secondoftwo%
                    }%
                }{%
                    \etb@dbg@fail{hsh}%
                    \@secondoftwo%
                }%
            }{%
                \etb@dbg@fail{tok}%
                \@secondoftwo%
            }%
        }{%
            \etb@dbg@fail{mac}%
            \@secondoftwo%
        }%
    }%
}%

\long\def\etb@ifpatchable@i#1{%
    \endgroup%
    \etb@dbg@init#1%
    \ifundef{#1}{%
        \etb@dbg@fail{def}%
        \@secondoftwo%
    }{%
        \etb@dbg@info{def}%
        \ifdefmacro{#1}{%
            \etb@dbg@info{mac}%
            \ifdefparam{#1}{%
                \etb@dbg@info{prm}%
                \etb@ifscanable{#1}{%
                    \etb@dbg@info{tok}%
                    \etb@dbg@info{pos}%
                    \@firstoftwo%
                }{%
                    \etb@dbg@fail{tok}%
                    \@secondoftwo%
                }%
            }{%
                \etb@dbg@info{prl}%
                \ifdefprotected{#1}{%
                    \etb@dbg@info{pro}%
                }{}%
                \etb@dbg@info{pos}\@firstoftwo%
            }%
        }{%
            \etb@dbg@fail{mac}%
            \@secondoftwo%
        }%
    }%
}


% syntax: \patchcmd[prefix]{cstoken}{search}{replace}{succ code}{fail code}
% [f1] - new prefix, default: exisiting prefixes
%  f2  - macro to patch
%  f3  - search text
%  f4  - replace text
%  f5  - succ code
%  f6  - fail code
\def\patchcmd{%
    \etb@dbg@trce\patchcmd%
    \begingroup%
        \@makeother\#% This must be done before the characters of the search and replace test are read to tokens.
    \etb@patchcmd@% closes the group
    % [new prefixes]{cstoken}{search}{replace}{succ code}{fail code} follows
    % These arguments are read with \catcode`\#=12.
    % These setting is revoked by closing the group right at the beginning of \etb@ifpatchable
}

% [f1] - new prefixes, default: exisiting prefixes
%  f2  - macro to patch
%  f3  - search text
%  f4  - replace text
%  f5  - succ code
%  f6  - fail code
\long\def\etb@patchcmd@{%
    \@testopt{\etb@patchcmd}{########1}%
}

% [[#1]] - new prefixes, default: exisiting prefixes
%   #2   - macro to patch
%   #3   - search text
%   #4   - replace text
%   f1   - succ code
%   f2   - fail code
\long\def\etb@patchcmd[#1]#2#3#4{%
    \etb@ifpatchable#2{#3}{% closes the group from \patchcmd
        \etb@dbg@succ{ret}%
        \begingroup%
            \edef\etb@resrvda{%
                % expands to code, that defines \etb@resrvda as the new version of the macro,
                % according to the result of \meaning<macro to patch>
                % ####1 - exisiting prefixes (can be empty)
                % ####2 - exisiting parameter text (can be empty)
                % ####3 - exisiting replacement text (can be empty)
                % The existing prefixes, parameter text, replacement text are not TeX tokens!
                % \etb@resrvda, the new prefixes, parameter text, replacement text are not yet TeX tokens!
                \def\noexpand\etb@resrvda####1\detokenize{macro:}####2->####3&{%
                    #1\def\string\etb@resrvda\space####2{\noexpand\etb@resrvdb####3&}%
                }%
                % when expanded, replaces the exisiting replacement text by the new
                % ####1 - replacement text before the search text (can be empty)
                % ####2 - replacement text after  the search text (can be empty)
                % All arguments are not TeX tokens!
                % The replacement text of this macro does not yet consists of TeX tokens!
                \def\noexpand\etb@resrvdb####1\detokenize{#3}####2&{%
                    ####1\detokenize{#4}####2%
                }%
                % expands to code, that simply defines \etb@resrvda as the new version of the macro
                % (The expansion of \meaning#2 is already part of the replacement text of this macro.)
                \edef\noexpand\etb@resrvda{%
                    \noexpand\etb@resrvda\meaning#2&%
                }%
            }%
            \etb@resrvda%
        \etb@patchcmd@scantoks\etb@resrvda% also closes the group
        \let#2\etb@resrvda%
        \undef\etb@resrvda%
        \@firstoftwo%
    }{%
        \@secondoftwo%
    }%
    % {succ code}{fail code} follows
}

% fully expands some code (maybe only a macro)
% and uses eTeX's \scantokens mechanism to: execute the expanded tokens under the current catcode regime.
% additionally closes the current group
% #1 - some code, will be fully expanded by \edef
%      mostly this is only one \macro, that will be fully expanded.
%      with \etb@patchcmd this code is of the form:
%           \etb@resrvda%
%      which fully expands to the simple definition
%           <prefixes>\def\etb@resrvda <parameter text>{<new replacement text>}
%      , where \etb@resrvda, <prefixes>, <parameter text>,<new replacement text> are not yet TeX tokens!
\def\etb@patchcmd@scantoks#1{%
    % define the procedure
    % extra macro is necessary to fix category codes for restoring
    % and have the full expansion of the the code in the argument present
    \edef\etb@resrvda{%
        \endgroup%
        \endlinechar\m@ne%
        \newlinechar\m@ne% patch by Sebastian Mesow: to keep newlinechars
        \unexpanded{%
            \makeatletter%
            \scantokens%
        }{#1}% argument of \scantokens is already fully expanded here.
        \endlinechar\the\endlinechar\relax%
        \newlinechar\the\newlinechar\relax% patch by Sebastian Mesow: to keep newlinechars
        \catcode\number`\@=\the\catcode`\@\relax%
    }%
    \etb@resrvda% execute the procedure
}

% \scantokes is like an eval command found in many programming languages.
% You can provide some text and it will be interpreted and executed as normal code.
% Here meaning, that the actually catcodes of the input is not relevant.
% But interpretion and execution is under the current catcode regime.

% {<cstoken>}{<code>}{<success>}{<failure>}

% f1 - macro to appto
% f2 - code to append
% f3 - succ code
% f4 - fail code
\def\apptocmd{%
    \etb@dbg@trce\apptocmd%
    \begingroup%
        \@makeother\#%
        \etb@hooktocmd\etb@append% {macro}{app code}{succ code}{fail code} follows
}

% f1 - macro to preto
% f2 - code to prepend
% f3 - succ code
% f4 - fail code
\def\pretocmd{%
    \etb@dbg@trce\pretocmd%
    \begingroup%
        \@makeother\#%
        \etb@hooktocmd\etb@prepend% {macro}{pre code}{succ code}{fail code} follows
}

% #1 - \etb@prepend or \etb@append
% #2 - macro to patch
% #3 - code to prepend or append
% f1 - succ code
% f2 - fail code
\long\def\etb@hooktocmd#1#2#3{%
    \endgroup%
    \etb@dbg@init#2%
    \ifundef{#2}{%
        \etb@dbg@fail{def}%
        \@secondoftwo%
    }{%
        \etb@dbg@info{def}%
        \ifdefmacro{#2}{%
            \etb@dbg@info{mac}%
            \ifdefparam{#2}{%
                \etb@dbg@info{prm}%
                \etb@ifscanable{#2}{%
                    \etb@ifhashcheck{#3}{%
                        \etb@dbg@info{tok}%
                        \etb@dbg@succ{ret}%
                        \etb@hooktocmd@i#1#2{#3}%
                        \@firstoftwo%
                    }{%
                        \etb@dbg@fail{hsh}%
                        \@secondoftwo%
                    }%
                }{%
                    \etb@dbg@fail{tok}%
                    \@secondoftwo%
                }%
            }{%
                \etb@dbg@info{prl}%
                \ifdefprotected{#2}{%
                    \etb@dbg@info{pro}%
                    \etb@dbg@succ{red}%
                    \protected%
                }{%
                    \etb@dbg@succ{red}%
                }%
                % Patch \macro without parameter
                \edef#2{%
                    #1{\expandonce#2}{\unexpanded{#3}}%
                }%
                \@firstoftwo%
            }%
        }{%
            \etb@dbg@fail{mac}%
            \@secondoftwo%
        }%
    }%
    % {succ code}{fail code} follows
}

% Patches \macro's that take parameters
% #1 - \etb@prepend or \etb@append
% #2 - \macroToPatch
% #3 - code to prepend or append
\long\def\etb@hooktocmd@i#1#2#3{%
    \begingroup%
    % defines a procedure to execute
    % This extra macro is needed, because the to prepend or append code (#3)
    % must be present \detokenize-ed at (re)definition time of \macroToPatch.
    \edef\etb@resrvda{% 1. version
        % reads the result of \meaning\macroToPatch (therefore the complicated parameter text)
        % ####1 - prefixes at the definition of \macroToPatch, like \long and \outer
        % ####2 - parameter text
        % ####3 - definition of \macroToPatch (= tokens when its is expanded once)
        % Then defines \etb@resrvda with the same parameter text as \macroToPatch (####2)
        % and this executes the original code and the prepended or appended code.
        \def\noexpand\etb@resrvda####1\detokenize{macro}:####2->####3&{% 2. version
            % will be the patched macro
            ####1\def\string\etb@resrvda\space####2{% 4. version
                #1{####3}{\detokenize{#3}}%
                % \etb@prepend or \etb@append with the original and the prepended or appended code is already expanded here
                % The original and to prepend or append code will not be expanded, because they were \detokeniz'ed.
            }%
        }%
        % At definition time of the following \macro the the result of \meaning\macroToPatch is provided.
        % At execution time the above described procedure is executed.
        \edef\noexpand\etb@resrvda{% 3. version
            \noexpand\etb@resrvda\meaning#2&% \meaning\macroToPatch is already expanded here.
            % The here the definition of \macroToPatch is provided.
            % The results of \meaning is just as the results of \detokenize.
        }%
    }%
    \etb@resrvda% executes the whole above described procedure
    % Now \etb@resrvda is not yet the new definition of \macroToPatch,
    % but when this version of \etb@resrvda is executed, then afterwards \etb@resrvda will be the new definition.
    % (Now \etb@resrvda has a speciaised definiton of the the 2. version of \etb@resrvda.)
    \etb@patchcmd@scantoks\etb@resrvda% This actually defines \etb@resrvda with the new definition of \macroToPatch.%
    % Therefore it retokenizes the \detokeniz-ed original and to prepend and append code under the current catcode regime.
    % It also closes the group. 
    \let#2\etb@resrvda% apply the patch
    \undef\etb@resrvda%
}

% executes the prepended code before the original code
% #1 - original code
% #2 - to prepend code
\long\def\etb@prepend#1#2{#2#1}

% executes the appended code after the original code
% #1 - original code
% #2 - to append code
\long\def\etb@append#1#2{#1#2}

\def\tracingpatches{%
    \etb@info{Enabling tracing}%
    \input{etoolbox.def}%
    \global\let\tracingpatches\relax%
}

\let\etb@dbg@trce\@gobble
\let\etb@dbg@init\@gobble
\let\etb@dbg@info\@gobble
\let\etb@dbg@succ\@gobble
\let\etb@dbg@fail\@gobble


\fi\endinput%