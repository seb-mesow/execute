% !TeX root = fsrprotocol3-cls

\ifx\latexplainGUARD\UNDEFINED
\def\latexplainGUARD{latex-plain.tex}

\catcode`\@=11% This is permanent

% This is a (re-)implementation of some few commands definied in latex.ltx to work under plain TeX

\def\makeatletter{%
    \catcode`\@11\relax%
}%

\def\makeatother{%
    \catcode`\@12\relax%
}%

% #1 – character
\def\@makeother#1{%
    \catcode`#1=12\relax%
}%

\long\def\@gobble#1{}%
\long\def\@firstofone#1{#1}
\let\@iden\@firstofone

\long\def\@gobbletwo#1#2{}%
\long\def\@firstoftwo#1#2{#1}%
\long\def\@secondoftwo#1#2{#2}%

% copied from latex.ltx
\def\two@digits#1{%
    \ifnum#1<10 % The space is important here!
        0%
    \fi%
    \number#1 % The space is important here!
}%

\def\three@digits#1{%
    \ifnum#1<100 % The space is important here!
        0%
    \fi%
    \two@digits{#1}%
}%

\def\four@digits#1{%
    \ifnum#1<1000 % The space is important here!
        0%
    \fi%
    \three@digits{#1}%
}%   

% copied from ltdefns.dtx
% #1 - token to test f1 against
% #2 - true code
% #3 - false code
% f1 - very next token to test against #1
\long\def\@ifnextchar#1#2#3{%
    \let\reserved@d=#1%
    \def\reserved@a{#2}%
    \def\reserved@b{#3}%
    \futurelet\@let@token\@ifnch% then comes f1
}%

% \reserved@d - token to test f1 against
% \reserved@a - true  code
% \reserved@b - false code
% \@let@token - arbitary very next token to test if it is equal to \reserved@d
\def\@ifnch{%
    \ifx\@let@token\@sptoken%
        \let\reserved@c\@xifnch%
    \else%
        \ifx\@let@token\reserved@d%
            \let\reserved@c\reserved@a%
        \else%
            \let\reserved@c\reserved@b%
        \fi%
    \fi%
    \reserved@c% what to do next: \xifnch or true code or false code
}%

\def\:{%
    \let\@sptoken= % note the space
}%
\:  % this makes \@sptoken a <space token>

\def\:{%
    \@xifnch%
}%
\expandafter\def\: {%
    \futurelet\@let@token\@ifnch%
}%
% equal to following pseudo code
% 
%     \def\@xifnch<space token>{%
%         \futurelet\@let@token\@ifnch%
%     }%
% 
% meaning \@xifnch gobbles a mandatory following space token and (re)starts \@ifnch

\let\kernel@ifnextchar\@ifnextchar%

% #1 - code which excepts an argument in brackets
% #2 - default value for that argument
\long\def\@testopt#1#2{%
    \kernel@ifnextchar[{%
        #1% call with provided optional argument in brackets which comes very next
    }{%
        #1[{#2}]% call with default value
    }%
}%

% copied from ltplain.dtx
\count10=22 % allocates \count registers 23, 24, ...
\count11=9 % allocates \dimen registers 10, 11, ...
\count12=9 % allocates \skip registers 10, 11, ...
\count15=9 % allocates \toks registers 10, 11, ...
\count16=-1 % allocates input streams 0, 1, ...
\count17=-1 % allocates output streams 0, 1, ...

% #1 - \csname for the type of register e.g. \dimen
% #2 - \csname that defines the \csname for the new register
% #3 - global counter, that counts all registers of that type
% #4 - normal   maximum count of registers of that type (-1, if only the extented maximum should be relevant)
% #5 - extented maximum count of registers of that type
% #6 - new \csname for naming the register
\def\e@alloc#1#2#3#4#5#6{%
    \global\advance#3\@ne%
    \e@ch@ck{#3}{#4}{#5}#1%
    \allocationnumber#3\relax%
    \global#2#6\allocationnumber%
    \wlog{\string#6=\string#1\the\allocationnumber}%
}%
% #1 - global counter
% #2 - normal maximum; if current global count is equal to this, jump the global count by 256
% #3 - extented maximum
% #4 - type \csname e.g. \dimen
\gdef\e@ch@ck#1#2#3#4{%
    \ifnum#1<#2 % The space is important here!
    \else%
        \ifnum#1=#2\relax%
        \global#1\@cclvi%
            \ifx\count#4%
                \global\advance#1 10 % The space is important here!
            \fi%
        \fi%
        \ifnum#1<#3\relax%
        \else%
            \errmessage{No room for a new \string#4}%
        \fi%
    \fi%
}%

\chardef\e@alloc@top=65535
\let\e@alloc@chardef\chardef
\let\float@count\e@alloc@top
\countdef\insc@unt=20
\countdef\allocationnumber=21

% remove \outer attribute from definitions by plain.tex
\let\newcount\@undefined
\let\newdimen\@undefined
\let\newtoks\@undefined
\let\newread\@undefined
\let\newwrite\@undefined

% f1 - control sequence for the new dimen register
\def\newdimen{\e@alloc\dimen\dimendef{\count11}\insc@unt\float@count}% \newdimencs follows
% f1 - control sequence for the new count register
\def\newcount{\e@alloc\count\countdef{\count10}\insc@unt\float@count}% \newcountcs follows
% f1 - control sequence for the new toks register
\def\newtoks{\e@alloc\toks\toksdef{\count15}\m@ne\e@alloc@top}% \newtokscs follows
% f1 - control sequence for the new read stream
\def\newread{\e@alloc\read\chardef{\count16}\m@ne\sixt@@n}% \newreadcs follows
\ifx\directlua\@undefined%
    % f1 - control sequence for the new write stream
    \def\newwrite{\e@alloc\write\chardef{\count17}\m@ne\sixt@@n}% \newwritecs follows
\else%
    % f1 - control sequence for the new write stream
    \def\newwrite{%
        \e@alloc\write{\ifnum\allocationnumber=18 % The space is important here!
                            \advance\count17\@ne%
                            \allocationnumber\count17 % The space is important here!
                        \fi%
                        \global\chardef%
        }{\count17}\m@ne{128}% \newwritecs follows
    }%
\fi%

\ifx\e@alloc@ccodetable@count\@undefined%
    \countdef\e@alloc@ccodetable@count=259 % The space is important here!
\fi%
% #1 - control sequence for the new catcodetable
\def\newcatcodetable#1{%
    \e@alloc\catcodetable\chardef\e@alloc@ccodetable@count\m@ne{"8000}#1%
    \initcatcodetable\allocationnumber%
}%
\e@alloc@ccodetable@count=\z@%



% copied from ltpar.dtx
\newdimen\@savsk%
\newcount\@savsf%

\def\@bsphack{%
    \relax%
    \ifhmode%
        \@savsk\lastskip%
        \@savsf\spacefactor%
    \fi%
}%

\def\@esphack{%
    \relax%
    \ifhmode%
        \spacefactor\@savsf%
        \ifdim\@savsk>\z@%
            \ifdim\lastskip=\z@%
                \nobreak%
                \hskip\z@skip%
            \fi%
            \ignorespaces%
        \fi%
    \fi%
}%

% leaves only the replacement text of a \meaning<macro>
% syntax: \expandafter\strip@prefix\meaning<macro>
\def\strip@prefix#1>{}

\def\@unexpandable@protect{%
    \noexpand\protect\noexpand%
}

\let\@typeset@protect\relax
\let\@not@protect\empty

\def\set@display@protect{%
    \let\protect\string%
}

\def\set@typeset@protect{%
    \let\protect\@typeset@protect%
}

\fi\endinput%
