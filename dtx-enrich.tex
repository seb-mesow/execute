% !TeX root = fsrprotocol3-cls

\ifx\dtxenrichGUARD\UNDEFINED
\def\dtxenrichGUARD{dtx-enrich.tex}

% This file implements the replacing (rp) mechanism for preprocessing .dtx files (or other files).
% It is implemented such, that it works under plain TeX **with eTeX extensions** (not LaTeX!)
% and that is loaded together with and after docstrip.tex.
% How to load:
% \begin{verbatim}
%     \input docstrip
%     \begingroup
%     \@@input dtx-enrich% docstrip.tex disencourages the use of the \input primitive.
%     \generate{...}% txs generate enriched .dtx files
%     \endgroup
%     \generate{...}% generate other files with normal docstrip
% \end{verbatim}

\input latex-plain % The space is important here!
\input etoolbox-plain % The space is important here!
\input execute % The space is important here!
               % needs LuaTeX !!!!!
\input latex-env % The space is important here!


% heavily inspired by \etb@ifscanable and \etb@patchcmd@scantoks from \pgk{etoolbox}
% TODO document this pattern somewhere
% #1 - \macro
% #2 - new replacement text; not yet as TeX tokens
\def\rp@scantokensto#1#2{%
    \edef\rp@reserved{%
        \def\string#1{#2}%
    }%
    \ltxenv@start%
    \scantokens\expandafter{\rp@reserved}% The current \catcode regime is relevant!
    % After that the \catcode's of the read chars are fixed (due to TeX's anatomy).
    \ltxenv@end%
    \undef\rp@reserved%
}%


\def\rp@remove@to@nil#1\rp@nil{}

% This should be so special, that it should not be used anywere else.
% If not, redefine the tags.
\def\rp@openingtag{<RP<}%
\def\rp@closeingtag{>RP>}%

% #1 – opening tag (already fully expanded)
% #2 – closing tag (already fully expanded)
\def\rp@defineParser#1#2{%
    % syntax: \processLine<content of one line>\endline
    % syntax: \processLine##1\endline
    % ##1 - content of one line of the processed file, can be empty
    \def\rp@processLine##1\endLine{%
        \let\rp@lines\empty%
        % strip everything upto the opening tag, if there is any
        \rp@readUntilOpeningTag##1#1\rp@nil%
    }%
    %
    % syntax: \rp@readUntilOpeningTag<content before the opening tag><opening tag><next token>
    % syntax: \rp@readUntilOpeningTag##1<opening tag><next token>
    % ##1 - content before the opening tag
    % f1  - next token; will NOT be gobbled from the input stream!
    \def\rp@readUntilOpeningTag##1#1{%
        \appto\rp@lines{##1}% process all the other stuff
        \futurelet\@let@token\rp@readUntilOpeningTag@i% <next token> follows
    }%
    %
    \def\rp@readUntilOpeningTag@i{%
        \ifx\@let@token\rp@nil%
            \expandafter\rp@endLine%
        \else%
            \expandafter\rp@readField%
        \fi%
    }%
    %
    % TODO handle unclosed opening tag
    \def\rp@readField##1#2{%
        \rp@processField{##1}%
        \rp@readUntilOpeningTag%
    }%
}
\edef\rp@reserved{%
    \noexpand\rp@defineParser{\rp@openingtag}{\rp@closeingtag}%
}
\rp@reserved

% #1 - field code, not yet as TeX tokens!
\def\rp@processField#1{%
\@bsphack%
    \rp@scantokensto\rp@fieldcode{#1}%
    \ltxenv@start%
    \let\@@protect\protect%
    \let\protect\@unexpandable@protect%
    \eappto\rp@lines{\execute{\rp@fieldcode}}%
    \let\protect\@@protect%
    \ltxenv@end%
\@esphack%
}
\def\rp@endLine{%
    \let\rp@doPerLine\rp@passTo@org@processLine%
%    \message{^^Jrl \detokenize\expandafter{\rp@lines}}%
    \expandafter\rp@doPerLine@process\expandafter{\rp@lines}%
    \rp@remove@to@nil%
}

\ltxenv@start
% #1 – newlinechar to use (already expanded)
\def\rp@define@doPerLine#1{%
    % executes \rp@doPerLine on each "line"
    % ##1 – "lines" separated by the newlinechar
    \def\rp@doPerLine@process##1{%
        \rp@doPerLine@readUntilNewLine##1#1\rp@nil%
    }%
    %
    % ##1 – "line" separated by the newlinechar
    % f1  – the newlinechar or \rp@nil
    \def\rp@doPerLine@readUntilNewLine##1#1{%
        \rp@doPerLine{##1}%
        \futurelet\@let@token\rp@doPerLine@readUntilNewLine@i% <next token> follows
    }%
    %
    \def\rp@doPerLine@readUntilNewLine@i{%
        \ifx\@let@token\rp@nil%
            \expandafter\rp@remove@to@nil%
        \else%
            \expandafter\rp@doPerLine@readUntilNewLine%
        \fi%
    }%
}
\rp@define@doPerLine{^^J}
\ltxenv@end

% #1 – a "line"
\def\rp@passTo@org@processLine#1{%
    \let\@@protect\protect%
    \let\protect\@unexpandable@protect%
    \rp@org@processLine#1\endLine%
    \let\protect\@@protect%
}

\let\rp@org@processLine\processLine%
\let\processLine\rp@processLine%

% start enviroment of user API field macros
\ltxenv@start%

% temporary fields
\newcount\rp@tmpcnta%
\newcount\rp@tmpcntb%
\newcount\rp@tmpcntc%

% adapted from plain.tex
\def\rp@loop#1\rp@repeat{%
    \def\rp@body{#1}%
    \rp@iterate%
}
\def\rp@iterate{%
    \rp@body%
        \let\rp@next\rp@iterate%
    \else%
        \let\rp@next\relax%
    \fi%
    \rp@next%
}
\let\rp@repeat=\fi% this makes \loop...\if...\repeat skippable

% TODO ideas for new field macros
% drivertemplate for the part with the <driver> guards

\def\fromoutfile{%
    \%\ \\iffalse^^J%
    \%\%\%\ From File: \protect\outfilename^^J%
    \%\ \\fi%
}%

% [f1] - seperator, default: "^^J\MetaPrefix\space"
\def\infilenames{%
    \@testopt{\infilenames@i}{^^J\MetaPrefix\space}%
}%

% [[#1]] - seperator
\def\infilenames@i[#1]{%
    % TODO pattern for \listConcat --> patch fp-lists
    \def\infilenames@seperator{% self-defining / self-redefining command
        % first call ONLY defines replacement text for further calls
        \def\infilenames@seperator{#1}%
    }%
    \let\infilenames@saved@do\do%
    \let\do\infilenames@do%
    \Name\iden{inwo@\outfilename}%
    \let\do\infilenames@saved@do%
}%

% This is macro can be redefined by users.
% Must store its results as plain text in \infilenames@result!
% #1 - infilename
% #2 - options
\def\infilenames@do#1#2{%
    \infilenames@seperator% add sep
    \ifstrempty{#2}{%
        #1%
    }{%
        #1\space(with Options: `#2')%
    }%
}%

% #1 - year  as a string representing an integer
% #2 - month as a string representing an integer
% #3 - day   as a string representing an integer
\def\rp@dateformat#1#2#3{%
    \four@digits{#1}/\two@digits{#2}/\two@digits{#3}%
}%

% The current date in LaTeX date format YYYY/MM/DD
\def\filedate{%
    \rp@dateformat{\the\year}{\the\month}{\the\day}%
}%

% to specifiy the minimum date of the LaTeX version to use at \NeedsTeXFormat{LaTeX2e}[<date>]
% The date is based on the current date.
% [#1] – offset to the current year , default: -1 
% [#2] – offset to the current month, default:  0 
% [#3] – offset to the current day  , default:  0
% For convenience it is assumed, that every month has 30 days (bank month).
\def\LaTeXdate{%
    \@testopt{\LaTeXdate@i}{-1}% [year offset][month offset][day offset] follows
}
\def\LaTeXdate@i[#1]{%
    \@testopt{\LaTeXdate@ii{#1}}{0}% [month offset][day offset] follows
}
\def\LaTeXdate@ii#1[#2]{%
    \@testopt{\offsetdate@iii{#1}{#2}}{0}% [day offset] follows
}

% to specifiy the minimum date of the version of a package to use at \RequirePackage{packagename}[<date>]
% The date is based on the current date.
% [#1] – offset to the current year ; default: -3
% [#2] – offset to the current month; default:  0 
% [#3] – offset to the current day  ; default:  0
% For convenience it is assumed, that every month has 30 days (bank month).
\def\packagedate{%
    \@testopt{\packagedate@i}{-3}% [year offset][month offset][day offset] follows
}
\def\packagedate@i[#1]{%
    \@testopt{\packagedate@ii{#1}}{0}% [month offset][day offset] follows
}
\def\packagedate@ii#1[#2]{%
    \@testopt{\offsetdate@iii{#1}{#2}}{0}% [day offset] follows
}

% specify an arbitray date as an offset to the current date
% [#1] – offset to the current year ; default: 0
% [#2] – offset to the current month; default: 0 
% [#3] – offset to the current day  ; default: 0
% For convenience it is assumed, that every month has 30 days (bank month).
\def\offsetdate{%
    \@testopt{\offsetdate@i}{0}% [year offset][month offset][day offset] follows
}
\def\offsetdate@i[#1]{%
    \@testopt{\offsetdate@ii{#1}}{0}% [month offset][day offset] follows
}
\def\offsetdate@ii#1[#2]{%
    \@testopt{\offsetdate@iii{#1}{#2}}{0}% [day offset] follows
}

\def\offsetdate@iii#1#2[#3]{%
    \rp@tmpcnta\year%
    \rp@tmpcntb\month%
    \rp@tmpcntc\day%
    \advance\rp@tmpcntc#3 % space important!
    \rp@loop%
        \ifnum\rp@tmpcntc<1 % space important!
            \advance\rp@tmpcntc30 % space important!
            \advance\rp@tmpcntb-1 % space important!
    \rp@repeat%
    \rp@loop%
        \ifnum\rp@tmpcntc>30 % space important!
            \advance\rp@tmpcntc-30 % space important!
            \advance\rp@tmpcntb1 % space important!
    \rp@repeat%
    \advance\rp@tmpcntb#2 % space important!
    \rp@loop%
        \ifnum\rp@tmpcntb<1 % space important!
            \advance\rp@tmpcntb12 % space important!
            \advance\rp@tmpcnta-1 % space important!
    \rp@repeat%
    \rp@loop%
        \ifnum\rp@tmpcntb>12 % space important!
            \advance\rp@tmpcntb-12 % space important!
            \advance\rp@tmpcnta1 % space important!
    \rp@repeat%
    \advance\rp@tmpcnta#1 % space important !
    \rp@dateformat{\the\rp@tmpcnta}{\the\rp@tmpcntb}{\the\rp@tmpcntc}%
}

\def\filetime{%
    % adapted from ltdirchk.dtx
    \rp@tmpcnta\time%
    \divide\rp@tmpcnta60 % The space is important here !
    \rp@tmpcntb=-\rp@tmpcnta%
    \multiply\rp@tmpcntb60 % The space is important here !
    \advance\rp@tmpcntb\time%
    %
    \two@digits{\the\rp@tmpcnta}:\two@digits{\the\rp@tmpcntb}%
}%

% end enviroment of user API field macros
\ltxenv@end%


\let\@fileX@saved\@fileX%
% redefinition!
% #1 -\chardef \<outfile>
% #2 -\from{<infilename>}{<options>}-code
\def\@fileX#1#2{%
    \let\curinwonames\empty%
    \@fileX@saved{#1}{#2}%
    \Name\edef{inwo@\@stripstring#1}{\curinwonames}%
%    \expandafter\infilenames@result@compute\expandafter{\@stripstring#1}%
}%
\let\@from@saved\@from%
% redefinition!
% #1      - infilename
% #2      - options
% \curin  - expands to the infilename (same as #1)
% \curout - expands to \chardef \<outfilename> as in \putline@do
\def\@from#1#2{%
    \@from@saved{#1}{#2}%
    \@addto\curinwonames{\noexpand\noexpand\noexpand\do{#1}{#2}}%
}%

\let\rp@org@StreamPut\StreamPut%
% redefinition!
% #1 - \chardef \<outfilename>
%          - e.g. if the <outfile name> is "foo.tex", then #1 is \foo.tex as one command.
%          - It was generated with \csname<outfilename>\endcsname .
%          - It is definied with \chardef .
%          - It holds the value of the output stream associated with <outfilename>.
%          - It has a value from 0 to 15.
% #2 – line to write to the outfile
\def\StreamPut#1#2{%
    \edef\outfilename{\@stripstring#1}%
    \let\@@protect\protect%
    \let\protect\@not@protect%
    \rp@org@StreamPut#1{#2}%
    \let\protect\@@protect%
}%

\nopreamble%
\nopostamble%

\fi\endinput
