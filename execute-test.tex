% !TeX encoding = UTF-8
% !TeX spellcheck = en_US

\hoffset\dimexpr-1in+5mm\relax%
\voffset\dimexpr-1in+5mm\relax%
\pagewidth210mm\relax%
\pageheight297mm\relax%
\hsize\dimexpr\pagewidth-10mm\relax%
\vsize\dimexpr\pageheight-10mm\relax%
\parskip1ex\relax%
\parindent0pt\relax%
\leftskip0pt\relax%
\rightskip0pt plus .5\hsize\relax

\tt\frenchspacing%

\input execute % The space is important!

\def\makeatletter{\catcode64=11 }% The space is important!
\def\makeatother{\catcode64=12 }% The space is important!

\makeatletter
\catcode`\_=12 % space important!

\def\section#1{%
    \def\tempa{========= #1 ==================}%
    \vskip1.5\baselineskip%
    {\tt\tempa\par}%
    \message{^^J\tempa}%
}%


\section{macro test}

\let\mydef\def
\def\todef{This is the WRONG text.}
\def\definer{\mydef\todef{This is the right text.}}

old register value: |\meaning\todef|

define env and execute: |\edef\env{\execute{\definer}}| =!= ||

new register value: |\meaning\todef|

meaning env: |\meaning\env| =!= |\meaning\empty|

expand env: |\env| =!= ||

new register value (again): |\meaning\todef|



\section{toks test}

\newtoks\mytoks
\mytoks{This are NOT the toks I want.}
\def\definer{\mytoks{This are the toks I want.}}

old register value: |\detokenize\expandafter{\the\mytoks}|

% WORKS:
define env and execute: |\edef\env{\execute{\definer}}| =!= ||
% WORKS TOO:
% define env and execute: |\edef\env{\immediateassigned{\definer}}|
% WORKS:
% define env and execute: |\edef\env{\execute{H\definer H}}|
% DO NOT WORKS:
% define env and execute: |\edef\env{\immediateassigned{H\definer H}}|

new register value: |\detokenize\expandafter{\the\mytoks}| =!= ||

meaning env: |\meaning\env| =!= |\meaning\empty|

expand env: |\env| =!= ||

new register value (again): |\detokenize\expandafter{\the\mytoks}|



\section{counts test}

\let\myadvance\advance
\newcount\mycnt
\mycnt=11\relax
\def\definer{\myadvance\mycnt11 }% The space is important here!

old register value:|\number\mycnt|

define env and execute: |\edef\env{\execute{\definer}}| =!= ||

new register value: |\number\mycnt| =!= |22|

meaning env: |\meaning\env| =!= |\meaning\empty|

expand env: |\env| =!= ||

new register value (again): |\number\mycnt| =!= |22|



\section{\string\@defaultunits\ test}

%%% From File: ltfsstrc.dtx
\def\remove@to@nnil#1\@nnil{}%

%%% From File: ltfssbas.dtx
% syntax: \@defaultunits<dimen register><length><default unit>\relax\@nnil
% When <length> has a  unit, then this unit           is used and the <default unit> is gobbled.
% When <length> has no unit, then the  <default unit> is used and nothing relevant   is gobbled.
\def\@defaultunits{%
    \afterassignment\remove@to@nnil%
}%

\newdimen\mylen

old register value: |\the\mylen|

define env and execute (1): |\edef\env{\execute{\@defaultunits\mylen1cmpt\relax\@nnil}}| =!= ||

new register value (1): |\the\mylen| >! |28pt|

meaning env (1): |\meaning\env| =!= |macro:->|

expand env (1): |\env| =!= ||

define env and execute (2): |\edef\env{\execute{\@defaultunits\mylen1pt\relax\@nnil}}| =!= ||

new register value (2): |\the\mylen| =!= |1.0pt|

meaning env (2): |\meaning\env| =!= |macro:->|

expand env (2): |\env| =!= ||



\section{\string\relax\ test}

\newcount\mycnt
\mycnt=11\relax
\def\definer{\myadvance\mycnt11\relax}

old register value:|\number\mycnt|

define env and execute: |\edef\env{\execute{\definer\relax}}| =!= ||

new register value: |\number\mycnt| =!= |22|

meaning env: |\meaning\env| =!= |\meaning\empty|

expand env: |\env| =!= ||

new register value (again): |\number\mycnt| =!= |22|



\section{\string\immediate\string\openout\ test}

\let\myimmediate\immediate
\let\myopenout\openout
\newwrite\auxout

\def\definer{%
    \myimmediate\myopenout\auxout\jobname.aux % The space is important here !; for testing scan_filename_text
}%

define env and execute: |\edef\env{\execute{\definer}}|

meaning env: |\meaning\env| =!= |\meaning\empty|

expand env: |\env| =!= ||



\section{\string\immediate\string\write\ test}

\let\mywrite\write
\def\command{this is command}
\def\a{this is a}
\def\w{written}
\def\writtenTextI{I'm text immediately \w\space to the .aux file with \noexpand\command\%\noexpand\a.}
\def\definer{%
    \myimmediate\mywrite\auxout{Lua: \writtenTextI}%
}

define env and execute: |\edef\env{\execute{\definer}}|

\myimmediate\mywrite\auxout{TeX: \writtenTextI}%

meaning env: |\meaning\env| =!= |\meaning\empty|

expand env: |\env| =!= ||



\section{\string\immediate\string\closeout\ test}

\let\mycloseout\closeout
\def\definer{%
    \myimmediate\mycloseout\auxout%
}%

define env and execute: |\edef\env{\execute{\definer}}|

meaning env: |\meaning\env| =!= |\meaning\empty|

expand env: |\env| =!= ||



\section{\string\openin\ test}

\let\myopenin\openin
\newread\auxin

\def\definer{%
    \openin\auxin{\jobname.aux}% for testing scan_general_text
}%

define env and execute: |\edef\env{\execute{\definer}}|

meaning env: |\meaning\env| =!= |\meaning\empty|

expand env: |\env| =!= ||



\section{\string\read\ test}

\let\myread\read
\def\definer{%
    \read\auxin to\lualine%
}%

define env and execute: |\edef\env{%
    \string\endlinechar=\the\endlinechar%
    \execute{\definer}%
}|

meaning lualine: |\meaning\lualine|

Note that TeX appends the current \string\endlinechar\ at the end of every read line.
TeX then converts (an \string\endlinechar\ which is) <return> == \string^\string^M == 13 to a space token.

meaning env: |\meaning\env| =!= |\meaning\empty|

expand env: |\env| =!= ||



\section{\string\readline\ test}

\let\myreadline\readline
\def\definer{%
    \readline\auxin to\texline%
}%

define env and execute: |\edef\env{%
    \string\endlinechar=\the\endlinechar%
    \execute{\definer}%
}|

meaning texline: |\meaning\texline|

Note that TeX appends the current \string\endlinechar\ at the end of every read line.
eTeX NOT converts the appended <return> == \string^\string^M == 13 to a space token.
As with other read characters, except the space, eTeX assigns catcode 12 == other to these characters.
And other characters are like letters simply typeset.
In the (O)T1 font encoding at slot/code point 13 == `015 with the Computer Modern Typewriter font cmttxx is the glyph of an apostrophe.

This code after this text in the test TeX file illustrates the issue.%
\catcode`\^^M=12 ^^M\catcode`\^^M=5 %
% The space above is important to end the number 5 and the assignment already on that line

meaning env: |\meaning\env|  =!= |\meaning\empty|

expand env: |\env| =!= ||



\section{\string\closein\ test}

\let\myclosein\closein
\def\definer{%
    \closein\auxin%
}%

define env and execute: |\edef\env{\execute{\definer}}|

meaning env: |\meaning\env| =!= |\meaning\empty|

expand env: |\env| =!= ||



\section{latex-env test}

\input latex-env

\def\todef{This is the FALSE text.}

\def\definer{\def\todef{This is the right text: percent <\%>, space <\ >, backslash <\\>}}

old register value:|\meaning\todef|
\par

\makeatletter
%\exec@printtokinfo{\%\ \\}
\par

\ltxenv@start

%\exec@printtokinfo{\%\ \\}
\par
define env and execute: |\edef\env{\execute{\%\ \\\definer}}|
%define env and execute: |\edef\env{\&\ \\}|
\par
new register value (\string\meaning): |\meaning\todef|
\par
Note that we \string\def'ine the macro \string\todef\ and not \string\edef'ine it.\space
Though the control symbols that are macros are not yet expanded.
\par
new register value (not \string\meaning): |\todef|
\par
meaning env: |\meaning\env|
\ltxenv@end
\makeatother

expand env: |\env|

new register value (again): |\meaning\todef|



\section{\string\filedate\ test}

\input latex-plain

\makeatletter

% #1 - year  as a string representing an integer
% #2 - month as a string representing an integer
% #3 - day   as a string representing an integer
\def\rp@dateformat#1#2#3{%
    \four@digits{#1}/\two@digits{#2}/\two@digits{#3}%
}%

\def\filedate{%
    \rp@dateformat{\the\year}{\the\month}{\the\day}%
}%

\def\todef{This is NOT the current date.}
\def\definer{\edef\todef{\execute{The current date is \filedate.}}}

old register value:|\meaning\todef|

define env and execute: |\edef\env{\execute{\definer}}|

new register value: |\meaning\todef|

meaning env: |\meaning\env|

expand env: |\env|

new register value (again): |\meaning\todef|

\section{\string\expandafter\ test}

\newcount\testcnt
\testcnt0 % space important!

\def\first{\advance\testcnt1 }%
\def\second{\multiply\testcnt2 }%

|\execute\expandafter{\expandafter\second\execute\expandafter{\first}}| =!= ||

\the\testcnt\ =!= 2

% TODO \begingroup \endgroup test

\section{\string\begingroup\ \& \string\endgroup\ test}

\let\A@token=A

\def\definer{%
    \begingroup%
    \futurelet\let@token\definer@i%
}
\def\definer@i{%
        \ifx\let@token\A@token%
            \aftergroup\@firstoftwo%
        \else
            \aftergroup\@secondoftwo%
        \fi%
    \endgroup%
    {(The next token is the letter "A".)}%
    {(The next token is NOT the letter "A".)}%
}

define env and execute: |\edef\env{\execute{\definer A}}|

\string\meaning\string\env: |\meaning\env|



\section{\string\bgroup\ \& \string\egroup\ test}

\def\definer{%
    \bgroup%
    \futurelet\let@token\definer@i%
}
\def\definer@i{%
        \ifx\let@token\A@token%
            \aftergroup\@firstoftwo%
        \else
            \aftergroup\@secondoftwo%
        \fi%
    \egroup%
    {(The next token is the letter "A".)}{(The next token is NOT the letter "A".)}%
}

define env and execute: |\edef\env{\execute{\definer B}}|

\string\meaning\string\env: |\meaning\env|



\section{remaining braces test}

\def\definer{%
    {argument text}%
}

define env and execute: |\edef\env{\execute{\definer}}|

\string\meaning\string\env: |\meaning\env|



\section{lots of text \& \string\loop\ test}

\newcount\loopcnt%
\loopcnt=0 % The space is important here!
|\execute{%
    \loop%
        This is many text.\space%
        \advance\loopcnt1 % The space is important here!
    \ifnum\loopcnt<40 % The space is important here!
    \repeat%
}|%

Note that the last space before | is the same space as between every two sentences.



\expandafter\section\expandafter{\string\par\ test}

define \string\todef\ and execute: |\edef\todef{\execute{first line\par second line}}|%

\string\meaning\string\todef: |\meaning\todef|

simply expand and execute \string\todef: |\todef|



\section{\string\protected\ tokens}
% TODO do not expand \protected tokens in \execute{}

\def\unprotectedTok{}
\def\protectedTok{}
\def\protect{\noexpand\protect\noexpand}

%define \string\env\ and execute: |\edef\env{\execute{\unprotectedTok\protectedTok}}| =!= ||
define \string\env\ and execute: |\edef\env{\execute{\unprotectedTok\protect\protectedTok}}| =!= ||

\def\compare{\protect\protectedTok}

%\string\meaning\string\env: |\meaning\env| =!= |\meaning\compare|
\string\meaning\string\env: |\meaning\env| =!= |\meaning\compare|



\section{test if group toks can be executed inside tex.runtoks()}

\begingroup
%\catcode`\_12 %
\catcode`\[1 %
\catcode`\]2 %
\catcode`\{12 %
\catcode`\}12 %

\directlua[
    left_brace_cmd  = token.command_id("left_brace" )
    right_brace_cmd = token.command_id("right_brace")
    left_brace_tok  = token.new(string.utfvalue("{"), left_brace_cmd )
    right_brace_tok = token.new(string.utfvalue("}"), right_brace_cmd)
]

\def\test[outside definition of \string\test]

\meaning\test

\directlua[ tex.runtoks( function() token.put_next(left_brace_tok) end ) ]

\def\test[inside definition of \string\test]

\meaning\test

\directlua[ tex.runtoks( function() token.put_next(right_brace_tok) end ) ]

\meaning\test


\endgroup

\bye
